%*******************************************************
% Abstract
%*******************************************************
%\renewcommand{\abstractname}{Abstract}

% There shall be a summary abstract of the thesis (of approximately 300 words)
% immediately following the Table of Contents page(s).

%\pdfbookmark[1]{Abstract}{Abstract}
% \addcontentsline{toc}{chapter}{\tocEntry{Abstract}}
\begingroup
\let\clearpage\relax
\let\cleardoublepage\relax
\let\cleardoublepage\relax

\addchap{Abstract}
% Short summary of the contents in English\dots~a great guide by
% Kent Beck how to write good abstracts can be found here:
% \begin{center}
% \url{https://plg.uwaterloo.ca/~migod/research/beckOOPSLA.html}
% \end{center}

A full factorial parametric study was carried out on a numerical model of hybrid heating system consisting of an air source heat pump and a gas boiler. This thesis aims to use the \texttt{Modelica} modelling language to determine an optimal bivalent parallel operation temperature window for the hybrid heating system. A two-storey, residential home was modelled, verified and validated against the reference home, and year long simulations were performed to optimise the temperature window along two metrics: reducing carbon emissions and reducing annual running costs. All combinations of seven bivalent temperatures from minus two Celsius to four Celsius and six cut-off temperatures from five Celsius to ten Celsius were examined in year-long numerical simulations. The \texttt{Modelica}-\texttt{EnergyPlus} co-simulation was carried out via the \texttt{Spawn of EnergyPlus} utility.

The best performing combination in the sensitivity analysis was a bivalent temperature of one Celsius, and  a cut-off temperature of five Celsius. This combination resulted the lowest carbon emissions and the lowest annual heating cost for the Irish climate studied---a temperate oceanic climate. The distribution of natural gas usage and electricity usage were very different to one another across the different temperature windows. Gas usage was greatest at bivalent temperatures of three and four Celsius. Electricity usage peaked at minus two Celsius and ten Celsius bivalent and cut-off temperatures, reaching a minimum at four Celsius and five Celsius bivalent and cut-off temperatures respectively. 

Combinations with higher bivalent and lower cut-off temperatures were more sensitive to gas and electricity price changes, most likely due to their higher energy consumption overall. Combinations with bivalent temperatures that allowed the heat pump to operate below the temperatures at with frosting occurred on the heat pump evaporator coils resulted in up to 50\% greater yearly on-off cycling rates. Seasonal coefficient of performance varied only slightly with varying bivalent operation window.

\bigskip\bigskip\bigskip\bigskip
\textbf{Keywords}:~\mykeywords.

% \vfill

% \begin{otherlanguage}{ngerman}
% \pdfbookmark[1]{Zusammenfassung}{Zusammenfassung}
% \chapter*{Zusammenfassung}
% Kurze Zusammenfassung des Inhaltes in deutscher Sprache\dots
% \end{otherlanguage}

\endgroup

% \vfill
