%************************************************
\chapter{Introduction}\label{ch:intro}
%************************************************


% Introduction should provide context to topic. "to be read by the CEO". Should include overall aim (and specific objectives), motivation, novelty, thesis layout, and most importantly, THE PROBLEM. No conclusions. c. 5 pages. 

\section{Context}

In Europe in 2022, the  residential sector was responsible for 27\% of final energy consumption \cite{}
Domestic water heating and space heating collectively account for close to 80\% of a household's energy usage. \cite{}

Climate change has also directly affected heating and cooling design. \citeauthor{owen_ashrae_2009} highlight that for \num{1274} weather stations/observing sites worldwide with sound data between 1974 and 2006, the averaged design conditions (which are explained in \cref{sec:hddanddesign}) over all locations had changed by the following:
\begin{itemize}
    \item the 
\end{itemize}

In Ireland, the housing stock increased by just 0.4\% between 2011 to 2016 \cite{cso_2020}. Very few new houses are being constructed with the possibility for newer, more efficient space heating and/or hot water production systems and better, holistic insulation. A similar sentiment has been noted in other Western European countries, making this not a localised issue, but rather an international one \cite{klein_numerical_2014, dongellini_influence_2021} .Thus, in order to reduce \ac{PE} consumption in any meaningful way, retrofits must be carried out on existing buildings. This includes adding insulation to attic spaces and/or walls of the house and the installation of more efficient heating systems. An advantage of \acs{HHS} is that existing buildings presumably already have a heat generator, be it a gas boiler or otherwise, which can be easily integrated into a \ac{HHS} with the addition of a \HP. Of course, plummbing works must be carried out and the \HP itself has a relatively high barrier to entry in the form of a high upfront cost. Currently the \citeauthor{seai_2020} do not give grants for the installation of \HPs as they do not deem them to be a renewable type of heat generator. This is partly true as \HPs do use electricity to run. 

The performance \acs{ASHP}, or \HPs in general, is very different to that of a traditional gas condensing boiler. The performance of a \HP is almost entirely determined by the outdoor temperature and climatic conditions. The performance of a \HP is described by the \ac{COP} of the unit. This measure varies troughout a heating season, day and even from minute to minute. A \HP with a \ac{COP} of 3 for example, produces three units of heat energy for every unit of electricity supplied. This \textit{extra} energy is being gathered from a renewable energy source --- which in the case of \acs{AWHP} is the external air. The amount of non-renewable energy consumed by \HP at any given time depends on the \ac{RES} of the grid. According to \citeauthor{seai_2020}, Ireland's \ac{RES} for electricity is around 9.3\%. This figure is epxected to increase in the coming years/decades as more wind turbines are installed, other renewable energy generators are built, the Celtic Interconnector subsea line between Ireland and France, and multiple non-renewable energy plants are decommisioned.

Since the \ac{COP} of a \ac{ASHP} varies quite drastically over a heating season, the measure \ac{SCOP} is often used to describe the performance of a \AWHP over a year or a heating season. 




\acs{HP} have over recent years become more popular throughout Europe 

There are three main types of \acs{HP} for space heating (i.e., not air-conditioning): \acs{AWHP}, Ground-Water Heat Pumps and Hydro-Water Heat Pumps. Ground-Water \acs{HP} acquire their heat energy by exploiting the heat contained within the Earth's soil. Soil, below a certain depth has a very consistent heat, only fluctating mildly seasonally. The added benefit of this type is that soil below a certain depth will not freeze, which would cause frosting like in \acs{AWHP}. Hydro-Water \acs{HP} gain their heat from water sources such as ponds, lakes or well-water. The temperature of water fluctuates far less than the ambient air temperature, meaning they do not extract as much energy as \acs{AWHP} on warmer days, howerver, during warmer days, the heating load of a residential home is much less than the peak load. Conversely, during very cold days, the water remains much warmer than the air, which is very beneficial during those high-load spells. These two types of \acs{HP}, due to their heat soruces, have their merits, however, it is also due to their heat sources that they are relatively obscure and not commonplace. Installing these types of \acs{HP} is costly, complicated, time consuming and require permits to build. Due to these reasons, \acs{AWHP} are the most common form of \ac{HP} sold in Europe \cite{epha_2015}.  

Frosting is detrminental to the performance of \acs{HP}. During cold, humid weather, frost builds up on the evaporator coils on the outdoor component of the \HP. Frosting dramatically lowers the heat conductivity between the coils and the ambient air, being essentially insulated by the frost. Frosting is a major concern in cool, humid climates, Ireland being one such climate.

A \ac{HHPS} as opposed to monovalent systems, is a configuration of a \HP in combination with a conventional gas boiler. During warmer days, the \HP has sufficient heating capacity to provide all the energy needed to heat a space, while being very efficient, while on colder days, it may be not economical or ecological to run the \HP. During these peroids, the majority of the heating load is passed to the gas boiler, which is not affected by the ambient air temperature. A control system can be put in place to intelligently turn on and off the \HP and gas boiler to better suit the current weather, for either economical or ecological reasons, or a weighted combination of the two. A (INSERT NAME FOR IT HERE) system is where the predefined external temperatures for turning on/off the \HP/boiler are not coincident. This creates a temperature range wherein the \HP and boiler are running simultaneously. This is the focus of this thesis: where lies the optimal crossover points for boiler-only operation, bivalent operation and \HP-only operation, specifically for the Irish climate. This research has been carried out for other cliamte types. The Irish climate is unique in that the temperature range (during the heating season) is quite narrow, the humidity is quite high almost all year round (especially on the west coast) and the temperature is quite quite mild.  

\section{Aim}
The aim of this thesis is to first, give an overview of the current state of research regarding \HPs and explain their operation including advantages, disadvantages, principle of operation and use cases. 

\section{Motivation}
The operation, control and performacnce of \acs{HHS} consisting of \acs{AWHP} and traditional gas boilers has been moderately studied in the literature. This type of heating system has been simulated and testing in-situ in countries such as China, Japan, Germany and other continental European countries, however, the research regarding efficient control of such a system in the Irish climate, namely a temperate oceanic climate, has not (or at the least only partially) been explored. Ireland has a very changeable and mild climate, but the characteristic of note is its consistently high humidity. Humidity and low temperatures are the bane of \HP operation and efficiency.

\section{The Problem} % Along with Novelty

\section{Thesis Layout}
\cref{ch:litreview} is a literatuve review of the operation of \HPs, \acs{HHS}, 