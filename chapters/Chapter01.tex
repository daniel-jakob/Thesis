%************************************************
\chapter{Introduction}\label{ch:intro}
%************************************************


% Introduction should provide context to topic. "to be read by the CEO". Should include overall aim (and specific objectives), motivation, novelty, thesis layout, and most importantly, THE PROBLEM. No conclusions. c. 5 pages. 

\section{Context}\label{sec:context}

Largely, throughout the developed world, it is clear that residential energy usage accounts for a large share of total energy use, and of that, space heating and \ac{DHW} production account for the majority of final energy use. In the USA, \ac{HVAC} energy use is 50\% of all building energy use and in China, \ac{HVAC} energy use is between 50\%--70\% of building energy use \cite{ieaorg_2018}. It is estimated that by 2050, two thirds of all residential buildings will have a form of air conditioning unit, further increasing these percentage shares. Alone in 2021, space cooling demand rose by 6.5\% \cite{ieaorg_2022}. \graffito{Of course, it must be noted that air conditioning naturally rose sharply in no small part due to the COIVD-19 pandemic and subsequent isolation rules in place in many parts of the world.}
In Europe in 2022, the  residential sector was responsible for 27\% of final energy consumption \cite{eurostat_final_2018}. Domestic water heating and space heating collectively account for close to 80\% of a household's energy usage in Europe. \cite{eurostat_energy_2020}.
All of this is to say, energy use due to \ac{HVAC} and \ac{DHW} production are high and are expected to continue rising. 

Climate change has directly affected heating and cooling design. \citelist{owen_ashrae_2009}{institution} highlight that for \num{1274} weather stations/observing sites worldwide with sound data between 1974 and 2006, the averaged design conditions (which are explained in \cref{sec:hddanddesign}) over all locations had changed by the following:
\begin{itemize}
    \item The 99.6\% annual dry-bulb temperature increased by \SI{1.52}{\celsius}
    \item The 0.4\% annual dry-bulb increased by \SI{0.79}{\celsius} 
    \item Annual dew point increased by \SI{0.55}{\celsius} 
    \item Heating-degree days (base \SI{18.3}{\celsius}) decreased by \SI{237}{\celsius\day} 
    \item Cooling degree-days (base \SI{10}{\celsius}) increased by \SI{136}{\celsius\day}
\end{itemize}\graffito{As of writing, continental Europe is experiencing ``the most extreme event ever seen in European climatology'' with a mid-winter heatwave \cite{pakalolo_2023}}
All of these changes the mentioned parameters point towards an increase in global temperatures. The effects of climate change are affecting how building cooling and heating design is carried out, due to the fact that cooling loads are, in general, becoming lower, while heating loads are generally increasing. Milder winters are allowing \HPs to be, ever so slightly more efficient throughout a heating season. Already, \acp{HP} have over recent years become more popular throughout Europe \cite{ehpa_2015,nowak_2018}, but this is more likely due to higher efficiencies and lower costs of newer models.


The so-called \textit{electrification of heat} has been supported in the EU for some time now due to seeking carbon emissions reductions and also security of supply, which, due to events on-going as of the writing of this thesis, has indeed become more of an issue than previously thought\ldots\ Electric heating devices such as \HPs convert electricity into heat, creating the sought after link between building heating and the electrical grid \cite{heinen_electricity_2016}. However, this link will not come without growing pains, as more buildings rely on the electrical grid to provide electricity for heating, the electrical demand grows. Due to the nature of heating demand and weather/climate which generally affects large areas and subsequently a large number of houses simultaneously, the electrical grid would be of course put under large strain when a particularly cold spell of weather hits an area. These great peaks in energy demand are a problem when it comes to electrical grid deployment, as the real-time balancing of the grid becomes an increasingly difficult job with the large variability of renewable energy production methods such as wind. \citeauthor {vuillecard_small_2011, thomasen_decarbonisation_2021} \cite{vuillecard_small_2011, thomasen_decarbonisation_2021} propose that \acp{HHS} could alleviate these very high energy demands from heating systems, could they manage to intelligently switch to primarily gas operation during peak energy demand periods. 


In Ireland, the housing stock increased by just 0.4\% between 2011 to 2016 \cite{cso_2020}. Very few new houses are being constructed with the possibility for newer, more efficient space heating and/or hot water production systems and better, holistic insulation. A similar sentiment has been noted in other Western European countries, making this not a localised issue, but rather an international one \cite{klein_numerical_2014, dongellini_influence_2021}. Thus, in order to reduce \ac{PE} consumption in any meaningful way, retrofits must be carried out on existing buildings. This includes adding insulation to attic spaces and/or walls of the house and the installation of more efficient heating systems. An advantage of \acp{HHS} is that existing buildings presumably already have a heat generator, be it a gas boiler or otherwise, which can be easily integrated into a \ac{HHS} with the addition of a \HP. Of course, plumbing works must be carried out and the \HP itself has a relatively high barrier to entry in the form of a high upfront cost. Currently \citelist{seai_2020}{institution} do not give grants for the installation of \HPs as they do not deem them to be a renewable type of heat generator. This is partly true as \HPs do use electricity to run, which, as discussed in \cref{sec:primaryenergy}, is generated mostly by non-renewable means in Ireland currently.  

Heat transfer from low to high temperature regions does not occur through normal thermodynamic means. Refrigerators are special devices used to achieve this, utilizing the refrigeration cycle, with the vapour-compression refrigeration cycle being the most commonly used. The reversed Carnot cycle is the most efficient but is only an idealized theoretical model. \HPs and air conditioners have the same components, and a single system can be used for both heating and cooling by adding a reversing valve to the hydronic circuit \cite{cengel_thermo_2020}.

The performance of \acp{ASHP}, or \HPs in general, is very different to that of a traditional gas condensing boiler. The performance of a \HP is almost entirely determined by the outdoor temperature and climatic conditions. The performance of a \HP is described by the \ac{COP} of the unit. This measure varies throughout a heating season, day and even from minute to minute. A \HP with a \ac{COP} of 3 for example, produces three units of heat energy for every unit of electricity supplied. This \textit{extra} energy is being gathered from a renewable energy source --- which in the case of \acp{AWHP} is the external air. The amount of non-renewable energy consumed by \HP at any given time depends on the \ac{RES} of the grid. According to \citelist{seai_2020}{institution}, Ireland's \ac{RES} for electricity is around 9.3\%. This figure is expected to increase in the coming years/decades as more wind turbines are installed, other renewable energy generators are built, the Celtic Interconnector subsea line between Ireland and France, and multiple non-renewable energy plants are decommissioned.

Since the \ac{COP} of an \ac{ASHP} varies quite drastically over a heating season, the measure \ac{SCOP} is often used to describe the performance of a \HP over a year or a heating season. The  \ac{SCOP} is an important tool for measuring the performance of heat pumps because it provides a standardised way to compare the efficiency of different systems. The measure of \ac{SCOP} and \ac{SPF} are quite similar in that they are both a ratio of the total electrical energy input to the total heat energy output of the \ac{HP}, however, \ac{SCOP} can also include other parts of the heating system 

Frosting is detrimental to the performance of \acp{HP} \cite{di_perna_experimental_2015}. During cold, humid weather, frost builds up on the evaporator coils on the outdoor component of the \HP. Frosting dramatically lowers the heat conductivity between the coils and the ambient air, being essentially insulated by the frost. Frosting is a major concern in cool, humid climates, Ireland being one such climate.

A \ac{HHPS} as opposed to monovalent systems, is a configuration of a \HP in combination with a conventional gas boiler. During warmer days, the \HP has sufficient heating capacity to provide all the energy needed to heat a space, while being very efficient, while on colder days, it may be not economical or ecological to run the \HP. During these periods, the majority of the heating load is passed to the gas boiler, which is not affected by the ambient air temperature. A control system can be put in place to intelligently turn on and off the \HP and gas boiler to better suit the current weather. An alternative-parallel bivalent system is where the predefined external temperatures for turning on/off the \HP/boiler are not coincident, as discussed in \cref{subsubsec:biv-parallelop}. This creates a temperature range wherein the \HP and boiler are running simultaneously. This is the focus of this thesis: where lies the optimal crossover points for boiler-only operation, bivalent operation and \HP-only operation, specifically for the Irish climate. This research has been carried out for other climate types. The Irish climate is unique in that the temperature range (during the heating season) is quite narrow, the humidity is quite high almost all year round (especially on the west coast) and the temperature is quite mild. \cref{fig:heatingloaddurationcurvebiv} shows a heating duration curve, which are explained in \cref{sec:hddanddesign}, with the bivalent temperature and cut-off temperature overlaid. The red region shows the number of hours of the year where the boiler is active, to the left of the cut-off point, the boiler is the sole heat source of the \ac{HHPS}, while to the right of the cut-off point and left of the bivalent point, the boiler and \ac{HP} are producing heat. Right of the bivalent point, the \HP is the sole heat producer. 

\begin{figure}[htb]
    \centering
    \includegraphics[width=1\linewidth]{tikz/hybridopdurationcurve.tikz}
    \caption{Hybrid operation representation via heating duration curve}
    \label{fig:heatingloaddurationcurvebiv}
\end{figure}

\section{Overview of Methodology}
Briefly, this thesis uses a numerical model of an Irish residential house (in \texttt{EnergyPlus}) coupled with a \ac{HHPS} model developed in \texttt{Modelica} to co-simulate year long experiments. The model of the dwelling was based off a real house in Belturbet, Co. Cavan, and experimental values measured from the dwelling in-situ, over the course of a year and a half, were used to validate the house and heating system model. 42 year long experiments were simulated, each with a different bivalent operation temperature window, i.e., a sensitivity analysis was carried out, varying two parameters, the bivalent temperature of the \HP, and the cut-off temperature of the boiler. The affects of varying this window were analysed from an ecological basis and an economic basis. 

\section{Aim}
The overall aim of this thesis is to create a \texttt{Modelica}-\texttt{EnergyPlus} co-simulation model of a reference residential building and \ac{HHS}. Through this model, the dynamics of shifting and dilating the hybrid operation temperature window will be analysed in order to optimise the \ac{HHS} along two dimensions: minimising $\text{CO}_2$ emissions and minimising annual heating costs. In other words, a parametric study with fixed-factor design will be carried out investigating the effects of varying the year-fixed hybrid operation temperature window, in annual simulations of the model. This analysis will first be carried out using the Irish market as a case study, and the results will be generalised to determine inflection points.

\section{Motivation}
The operation, control and performance of \acp{HHS} consisting of \acp{AWHP} and traditional gas boilers has been moderately studied in the literature. This type of heating system has been simulated and tested in-situ in countries such as China \cite{li_parallel_2018}, Japan and Korea \cite{jang_continuous_2013, park_performance_2014}, North America \cite{rauschkolb_cost-optimal_2020, }, Germany \cite{klein_numerical_2014} and other continental European countries \cite{bagarella_annual_2016, roccatello_analysis_2022, amirkhizi_cost_2020, dongellini_influence_2021,di_perna_experimental_2015}, however, the research regarding efficient control of such a system in the Irish climate, namely a temperate oceanic climate, has not (or at the least only partially) been explored \cite{heinen_electricity_2016}. Ireland has a very changeable and mild climate, but the characteristic of note is its consistently high humidity. Humidity and low temperatures are the bane of \HP operation and efficiency.

By achieving the aims mentioned previously, this thesis aims to contribute to the development of energy-efficient and cost-effective \acp{HHS} that have the potential to reduce \ac{GHG} emissions and promote sustainable energy practices in the residential sector. The use of a rigorous and validated simulation model will enable a more accurate analysis of the system dynamics, allowing for the identification of optimal operational parameters and the potential for further optimisation in future research.

\section{Research Question} 
Research question: How can a year-static hybrid operation temperature window of a \ac{HHS} be optimised to minimise both $\text{CO}_2$ emissions and annual heating costs in a typical Irish residential home.

\section{Thesis Layout}
The thesis is structured as follows: 

\cref{ch:litreview} provides a literature review on \HPs,  \acp{HHS}, \acp{HDD}, \ac{PE}, electrification of heating, controllers and control theory, and verification and validation of models.

\cref{ch:method} outlines the methodology used, including the experimental reference building, building and heating system models, sensitivity analysis design, and eco-economic assessment. 

\cref{ch:method} focuses on the system model, including the location, form and fabric of the building, schedules, equipment and internal gains, and heating system. 

\cref{ch:sensitivity} presents the results of the parametric study, specifically the parametric study design, energy consumption (electricity and gas), and performance indices. 

\cref{ch:eco-ecoass} covers the eco-economic assessment, including analysis of the minimisation of the cost of annual heating, minimisation of $\text{CO}2$ emissions and optimising \ac{PES}.

Finally, \cref{ch:conclusions} presents the conclusions drawn from the study, as well as giving outlines for potential future work on the subject.
