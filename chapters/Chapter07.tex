%************************************************
\chapter{Conclusions}\label{ch:conclusions} 
%************************************************
%Did the models bring any thing of value?
%Intervals...
%Improvements... Future Work

\section{Conclusions}


\section{Future Work}
There are many avenues to pursue in regard to potential future work in the \ac{HHPS} field. Although a great effort was made to investigate the issue of varying bivalent temperature operation windows, there is always more to study, science is never finished. A brief list of areas of potential future research which could be fruitful includes:
\begin{itemize}
    \item Improve the frosting model implemented in this thesis. The frosting model described in \cref{subsubsec:frostingmod} is perhaps not the most developed or accurate model, a more nuanced model may include a model which tracks ice build up as a function both temperature and r.h. The data required for this may be too sparse to accurately model this behaviour however.
    \item Different constructions, perhaps a model representing a dwelling having undergone a deep retrofit could be investigated. The resulting improved thermal properties of the house would result in less heating demand overall and would surely alter the dynamics of the boiler-\HP relationship
    \item Calibration was only performed for one of the temperature windows, namely a bivalent temperature of \qty{7}{\celsius} and a cut-off temperature of \qty{2}{\celsius}. It would better if a second operation window was experimentally measured and calibrated with. 
    \item More climates could be investigated. The Irish climate is perhaps under researched in the field of \ac{HHPS}, however, there are also other climates which could be researched to further generalise the optimisation of the operation window.
    \item \ac{DHW} production was omitted in the simulation of this thesis, however it could be of merit to inquire into how  \ac{DHW} production would alter the dynamics of the \HP and boiler, given the boiler were not the sole producer of hot water.
    \item Scrutinise whether parallel vs. in-series \HP and boiler makes a difference in the analysis.
    \item Finally, smart, real-time bivalent operation temperature window shifting and dilating could be a very interesting topic to pursue. 
\end{itemize}

