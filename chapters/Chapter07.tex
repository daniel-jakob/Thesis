%************************************************
\chapter{Conclusions}\label{ch:conclusions} 
%************************************************
%Did the models bring any thing of value?
%Intervals...
%Improvements... Future Work


% \begin{algorithm}[htpb]
%     \SetKwFunction{counter}{counter}
%     \caption{Newton-Raphson Method}
%     \label{alg:newton-raphson}
%     \SetAlgoLined
%     \KwIn{Estimate $x_0$ and function $f(x)$}
%     \KwResult{Approximation of the root of $f(x)$}
%      \counter $\leftarrow$ 0\tcc*[r]{initialisations}
%      $x_n \leftarrow x_0$\;
%      \While{\counter\ $< 20$}{
%       $\displaystyle x_{n+1}=x_n-\frac{f\kern-0.25em\rbk{x_n}}{f'\kern-0.25em\rbk{x_n}}$\tcc*[r]{see: \cref{fig:fds}}
%       \If(\tcc*[f]{break func. if tol. level met}){$\abs{f\kern-0.25em\rbk{x_{n+1}}}<10^{-10}$}{Terminate\;
%        }
%        set $x_n$ to $x_{n+1}$\;
%        \counter = \counter + 1\tcc*[r]{increment counter}
%      }
%      \Return{$x_{n+1}$}
%     \end{algorithm}

\section{Conclusions}

\section{Future Work}
There are many avenues to pursue in regard to potential future work in the \ac{HHPS} field. Although a great effort was made to investigate the issue of varying bivalent temperature operation windows, there is always more to study, science is never finished. A brief list of areas of potential future research which could be fruitful includes:
\begin{itemize}
    \item Improve the frosting model implemented in this thesis. The frosting model described in \cref{subsubsec:frostingmod} is perhaps not the most developed or accurate model, a more nuanced model may include a model which tracks ice build up as a function both temperature and r.h. The data required for this may be too sparse to accurately model this behaviour however.
    \item Different constructions, perhaps a model representing a dwelling having undergone a deep retrofit could be investigated. The resulting improved thermal properties of the house would result in less heating demand overall and would surely alter the dynamics of the boiler-\HP relationship
    \item Calibration was only performed for one of the temperature windows, namely a bivalent temperature of \qty{7}{\celsius} and a cut-off temperature of \qty{2}{\celsius}. It would better if a second window was experimental measured and calibrated with. 
    \item More climates could be 
    \item DHW production
    \item See if parallel vs in-series \HP and boiler makes a difference
    \item Smart bivalent operation temperature window shifting and dilating. 
\end{itemize}

