%************************************************
\chapter{Sensitivity Analysis}\label{ch:sensitivity} 
%************************************************
% establish the thermodynamic parameters of the system. Effect of ambient air temp. establish behaviour of system in different climates (maybe). 
% How does the T_biv and T_cutoff temperature window affect raw energy consumption values. 

A sensitivity analysis, also referred to as a parametric study, is a technique utilised in simulation modelling to assess the impact of varying parameters on the outcomes of the model. The process involves iteratively running the simulation multiple times with systematic modifications to input parameters and can enable the identification of critical parameters that exert the most substantial influence on the model outcomes, and also, in particular, identify the optimal pair of values of the parallel operation temperature window. 

In this thesis, an exhaustive or full factorial sensitivity analysis has been performed on the bivalent parallel operation temperature window of a \ac{HHPS}. The analysis evaluates the impact of altering the bivalent temperature and cut-off temperature on the total cost of fuel and electricity for the \ac{HHS} over the course of a year in the economic assessment, as well as on the \ac{PE} used and $\text{CO}_2$ emissions in the ecological assessment. The approach adopted in this research allows for a comprehensive examination of the system, by identifying the factors that are critical in determining the optimal operation of the system, while also providing insights into potential cost savings and environmental benefits.

Although full factorial designs can be computationally expensive and time-consuming, this approach was chosen as the dimensionality of the parameters pace is very low at just two, and the resolution, or number of levels, being simulated for the two parameters is also relatively low. This enables a full factorial parametric study to b e carried out given the complexity of the model and the computational resources available.  