%*****************************************
\chapter{Literature Review}\label{ch:litreview}
%*****************************************
%todo: find relevant and insightful papers on the topic. FIN

%Hybrid heat pump can also refer to a dual energy input heat up. Find way to exclude this from search term. On google scholar, -"blahblah" can be used. Not sure if the same is true on OneSearch. 

% Very vert briefly explain how heat transfer work w.r.t. heating/cooling houses. Briefly explain heat pumps, some equations. Explain what a bivalent heat pump system is, will not be too commonly known. 
% Introduce relevant papers. What did they find, what did they do.

\section{Overview}
\acs{HP} work by harnessing the energy from low temperature sources such as air, water or the ground. 
Under ideal conditions, \acs{AWHP} have extremely high \acs{COP} in the \num{3.5} to \num{4.5} range. This is of course from their ability to harvest the aerothermal energy from the outside air. The main downfall of \acs{AWHP} is that when the external air temperature is low, their \ac{COP} is reduced significantly. Due to this inherent disadvantage, \acs{HP} are essentially unfit be the sole space heating generator for almost all applications, depending on climates and design points. While \HPs have the capacity to perform heating and cooling, this thesis and associated simulations do(es) not consider the cooling of a building or home, and therefore is only concerned with heating-seating heating. The space-heating radiators found in exisitng homes are not suitable for cooling \cite{klein_numerical_2014}, the cold water in the radiators does not cool the room effectively and condensation on the radiator surface may become and issue. 

Because the efficiency of \acs{HP} is so dependent on the outside air temperature, the measure of \ac{SPF} is typically used to characterise them when considering the performance over a certain heating peroid. The \ac{SPF} represents the ratio of the total usefull energy produced by the \ac{HP} during a heating season, to the seasonal electricity consumption. For example, an \ac{SPF} of 3 would mean that over a given year, the \ac{HP} produced 3 units of heating energy for every unit of electrical energy provided. Due to \acs{HP} extracting renewable energy from the surrounding air, the \ac{SPF} is (or should be) always higher than 1, and generally is above 3. EU legislation states that in order to be eligible for the \ac{RHI}, a \acs{HP} \ac{SPF} must be above 2.5 \cite{eu-114-2014}. 

\acs{HP} come in many different heat capacities, from single kilo-watt units to extremely large units which can heat large multi storey office buildings. In residential home contexts, the largest \acs{HP} generally available are almost \SI{300}{\kilo\watt}. If an \ac{ASHP} were to be sized so large as to have the capacity to provide the entire heating envelope of a residential home during even the coldest expected temperatures, the \ac{ASHP} would (aside from being prohibitively expensive), be so oversized that when temperatures are moderate, the \ac{HP} would produce so much heat as to heat the space so quickly that it would have an extremely short on-off cycle \cite{bee_air-source_2019}. Since the peak load for heating occurs for a very small number of hours during any given heating period, this would be very detrminental to the unit, specifically the condenser component. The frequent on-off cycling significantly reduces the longevity of the condenser, and would require replacement long before what would be expected. Many manufacturers suggest that the number of on-off cycles should not exceed 6 per hour. %FIND SOURCE
To avoid this issue, \acs{AWHP} are specifically undersized. Various ``design temperatures'' can be calculated for a given location. For Dublin, the design temperature which covers 99.0\% of the annual heating is \SI{-0.7}{\celsius}. \acs{AWHP} are usually sized to meet a design temperature of 60\%--70\%, as opposed to more traditional space heaters, as is fruther explained in \cref{sec:hddanddesign}% COME BACK TO THIS.

\acs{HP} tend to perform better when providing space heating through underfloor heating. This is partly due to underfloor heating being more efficient in general than other, more traditional space heating methods, namely hot-water radiators. Another reason more applicable to \acs{HP} is that the (space heating) inlet water temperature for underfloor heating is much lower than radiators. This means the \ac{HP} does not have to heat the circulating water as hot as it would with radiators. The temperature delta between water temperature inlet to the \ac{HP} and the outlet is simply lower and therefore less energy has to be produced by the \ac{HP} in the first place. However, retrofitting houses with underfloor heating is expensive and very intrusive to the building --- as obviosuly (all) floors much be ripped up and coils must be placed and plumped --- which discourages many homeowners from peforming this type of retrofit.

\section{Heat Pump}
\cite{klein_numerical_2014} investigated the efficiency gains from the implementation of a \ac{HHS} in a retrofitted house built in the 1970s. They compared this to other monovalent heating systems. A medium sized heat pump was used. 




\subsection{Buffer Tank} 
A buffer tank is a medium- to large-sized water vessel used in hydronic heating systems. It provides a large thermal inertia to the heating system-house system, which many small- to medium-sized houses, especially those with poor insulation, lack. Thermal inertia is a desired property of a building as rapid thermal fluctutations in ambient air are less of a concern when it comes to maintaining a comfortable thermal environment indoors. This effect is noticable in large office/district buildings with high thermal inertias and plays a significant role in heating-capacity selection \cite{owen_ashrae_2009}. Furthermore, a buffer tank provides a ``hydraulic switch'' and allows for heat generation and heat distribution to be in separate loops. This opens up the option to have differing flowrates between the heat generation and heat distribution loops. 

Buffer tanks have been found, when sized correctly and with an appropriate control strategy, to have a positivie influence on the efficiency and performance on \acs{HHS} \cite{klein_numerical_2014,roccatello_analysis_2022}. The controller is able to make use of the \HPs ``most profitable working conditions'' thanks to the presence of the buffer \cite{dettorre_economic_2018}. It has been found that when a buffer thank is present in the \HP circuit, \ac{SPF} increases as the size of the \HP decreases \cite{mugnini_variable-load_2021}. \citeauthor{mugnini_variable-load_2021} confirmed this for all sizes of \HPs simulated, the smallest buffer tank having a capacity of \SI{200}{\liter}. 

The larger a buffer tank in volume, the larger its energy storage capacity. However, with a larger volume, and naturally larger cylinder and surface area, comes greater heat loss, which seem to correlate almost linearly \cite{klein_numerical_2014}. This could be justified if other performance factors such as \ac{SPF} or load factor were positively affected to offset this loss in heat, however this does not seem to be the case according to \cite{roccatello_analysis_2022} and \citeauthor{klein_numerical_2014}, which also found only a moderate reduciton in on-off cycles with smaller tanks. This is partly to do with the thermal interia of the building and return temperature controller. 
\citeauthor{klein_numerical_2014} found that the volume of the buffer tank had very limited effect on the system performance. \citeauthor{dongellini_influence_2021} sized their buffer tank just large enough such that the maximum nummber of on-off cycles was never greater than six per hour,  resulting in a buffer tank with a volume of \SI{79}{\litre}.  This maximumm on-off cycle figure was chosen based off their \HP  manufacturer guidelines. Daiken suggest %FIND SUGGESTED MAX NUMBER OF ON-OFF CYCLES FROM DAIKIN. 
\cite{dettorre_economic_2018}

\subsection{Frosting and Defrosting}
Frosting occurs in \acs{ASHP} in colder ambient temperatures, typically from \qtyrange{-15}{6}{\celsius} \cite{sandstrom_frosting_2021}, resutling in issues due to the frost build up. This specifically occurs when the surface temperature of the fins on the air-side heat exchanger are lower than the dew point of the of the air. Water droplets start to form and collect on the fins. When the temperatures is below freezing or close to it, the water droplets freeze to the fins and build up a frosting. Frost, unlike snow, which both form from the freezing of water droplets, is not loose and must be scraped off or melted off. It will not \textit{fall off of} a surface like snow might. This layer of frost acts as a layer of insulation and restricts the heat exchanger from trannsferring heat from the ambient air. Since these fins are typically closely packed, if the layering of frost continues and progressively builds up, the airflow around the fins decreases and so does convective heat transfer to the ambient air, further exacerbating the issue of insulation. All of this is to say that when frosting occurs in \acs{ASHP}, their performance declines severely. \cite{zhang_experimental_2018} found that the temperature of the air and surface of the fins, humidity, velocity of air are the main factors involved in frost formation. 

Many treatments for frosting have been proposed and implemented into products. There is however no golden bullet solution, all of their advantages and disadvantages. Three main solutions are typically used when addressing the issue of frosting in \acs{ASHP}. 
\begin{itemize}
    \item Simple on-off defrosting: the \HP is simply switched off when too much frost has formed on the outdoor component. The performance has been degraded to such a point that it is now economically advantageous to turn off the \HP and wait for the frost to melt away. This however, takes a long time and can negatively affect the thermal comfort of a home if no other heat production is used. The \HP does not use any power during this off-cycle of course, retaining the \ac{COP} of the \HP --- although, this may affect the overall system performance if a gas boiler needs to be used to provide the entire heating load of the home.
    \item Reverse cycle defrosting: this method is similar to the first method; the refrigerant is cycled in reverse and hot gas is forced into the heat exchanger. Recall that \HPs and refrigerators differ only in objective. The \HP now treats the outdoors as the ``cold'' sink and begins transferring heat from indoors to outdoors. Intuitively, one can see that this is quite detrminental to the performance of the \HP as the house is being actively cooled in order to heat up the outdoor coils and fins to melt away the frost, and one cannot forget water's high thermal capacity... The intention in this method is to melt the frost much quicker than the first method, allowing the \ASHP to being warming the home once again much earlier than the the simple on-off defrosting method.
    \item Resistive heating: electric resistive heaters are installed on/in the heat exchanger. This method works very well, quickly melting off frost and is a separate heating element to the \HP and therefore does not interrupt the \HPs cycles. Resistive heaters are very expensive to run and negatively affect the \ac{COP} of the \HP.
\end{itemize}

\cite{amer_review_2017} found that the reverse cycling method resulted in a higher \ac{COP} than the other two methods. 

\section{Heating Degree Days and Design Temperatures} \label{sec:hddanddesign}
\acs{HDD} is a measure of the difference between the outside temperature and the inside temperature. \acs{HDD} are usually considered over a period of time, be it a month, heating season or entire year. A \textit{base} temperature is chosen, typically around \qtyrange{12}{21}{\celsius} which then determines when it is ``cold'' outside, or can be thought of as being the temperature above which heating is no longer considered to require heating. This base temperature can be chosen at will, and simply depends on what the person/institution deems to be \textit{warm enoughh}. This measure can be used to quantitatively compare the heating demand of a given house in different locations/climates. The heating requirement of a specific building are directly proportional to the \ac{HDD} \cite{chartered_institution_of_building_services_engineers_environmental_2006}.

To calculate  the \ac{HDD} for a certain day, three equations are used and are displayed from \crefrange{eq:hdd1}{eq:hdd3}. Which equation to use is determined by the interaction between the base temperature and the maximum temperature recorded during that day.  
\begin{equation}
    \text{Degree days} = \begin{cases}
        t_\text{base} - \frac12(t_\text{max} + t_\text{min}), & \text{if } t_\text{max} < t_\text{base} \label{eq:hdd1}\\
        \frac12(t_\text{base} - t_\text{min}) -\frac14(t_\text{max} -t_\text{base} ), & \text{if } t_\text{base} > \frac12(t_\text{max} + t_\text{min}) \\
        \frac14(t_\text{base} -t_\text{min} ), & \text{if } t_\text{base} <\frac12(t_\text{max} + t_\text{min}) %\label{eq:hdd3}
     \end{cases}  
\end{equation}

To co calculate the Monthly degree days however, only \cref{eq:hdd1} is made use of. This total is found by summing the daily temperatures differences and can be seen in \cref{eq:mdd}.
\begin{equation}
    \text{Montly degree days} = \displaystyle\sum_\text{month} \left[t_\text{base} - \frac12(t_\text{max} + t_\text{min})\right] 
\end{equation}

\citetitle{chartered_institution_of_building_services_engineers_environmental_2006} has chosen a base temperature of \SI{15.5}{\celsius}. \citetitle{owen_ashrae_2009} used a base temperature of \SI{18.3}{\celsius} and determined an annual \ac{HDD} of \SI{3135}{\celsius\day} for Dublin Airport, IE, N\ang{53;26;} W\ang{06;15;}. Using the online tool \texttt{Degree Days.Net} \cite{degreedays} with a base temperature of \SI{15.5}{\celsius}, a \ac{HDD} figure of \SI{2072.3}{\celsius\day} was obtained for the same location.  

Design temperatures are a measure how many hours/days a specified condtion is exceeded. In the case of a heating design temperature, this would indicate how many days of the year or heating season are spent below a given temperature. \citetitle{owen_ashrae_2009} notes that this measure does not give an indication of the freqeuncy or duration of these events, only a cummulative result is returned. According to \citetitle{owen_ashrae_2009}, the 99.6\% design temperature in Dublin Airport is \SI{-1.9}{\celsius} while the 99.0\% design temperature is \SI{-0.7}{\celsius}. Traditionally, conventional gas boilers or resistive heaters were sized to design temperatures, meaning, for a chosen design temperature percentile (e.g., 99.0\%), the heater could heat the building to thermally comfortable levels for 99\% of the year, however during the 1\% temperature lows, the heater would not be adaquate. This calculates to the heater being undersized for $\sim$35 hours of the year. \graffito{$365\times24=\SI{8860}{\hour} \Rightarrow 99.0\%-ile = 8760(100-99.0) = 87.6$ }

In monovalent systems, the \HP is sized in such a way as to be able to provided the entire heating load for a building at design conditions. This results in the \HP being positively overdimensioned for the task \cite{klein_numerical_2014}. 

The concept of a \textit{design-day} can be used to design heating configurations for homes, especially when performing numerical simulations on a model of the system \cite{rauschkolb_cost-optimal_2020}. A design-day file is a special weather file created with design conditions in mind. Based on the design temperature parameter, \citelist{ashrae_2009}{institution}, lays out a procedure to generate a 24-hour weather profile. These profiles represent the 0.4\% to 99.6\% extremes experienced for a particular location. This weather data is used in simulations to determine the minimum size for a heater required for a house (for these particular percentiles of course). \graffito{for the purposes of the simulation(s) concerning this thesis, the 0.4\%-ile, and any cooling-nessecary-temperatures for that matter, are not of concern as cooling is out of scope.}

The ``heating duration curve'' can be devised for a specific building for a specific climate. This chart plots the number of hours 
